%-----------------------------------
% Define document and include general packages
%-----------------------------------
\documentclass[12pt,oneside,titlepage,listof=totoc,bibliography=totoc]{scrartcl}
\usepackage[utf8]{inputenc}
\usepackage[ngerman]{babel}
\usepackage[babel,german=quotes]{csquotes}
\usepackage[T1]{fontenc}
\usepackage{fancyhdr}
\usepackage{fancybox}
\usepackage[a4paper, left=4cm, right=2cm, top=3cm, bottom=2cm]{geometry}
\usepackage{graphicx}
\usepackage{colortbl}
\usepackage{array}
\usepackage{float}      %Positionierung von Abb. und Tabellen mit [H] erzwingen
\usepackage{footnote}
\usepackage{caption}
\usepackage{enumitem}
\usepackage{amssymb}
\usepackage{mathptmx}
\usepackage{amsmath}
\usepackage[table]{xcolor}
\usepackage{marvosym}			% Verwendung von Symbolen, z.B. perfektes Eurozeichen
\usepackage[colorlinks=true,linkcolor=black]{hyperref}
\definecolor{darkblack}{rgb}{0,0,0}
\hypersetup{colorlinks=true, breaklinks=true, linkcolor=darkblack, menucolor=darkblack, urlcolor=darkblack}
\fontfamily{ptm}\selectfont

% Mehrere Fussnoten nacheinander mit Komma separiert
\usepackage[hang, multiple]{footmisc}
\setlength{\footnotemargin}{1em}

% todo Aufgaben als Kommentare verfassen für verschiedene Editoren
\usepackage{todonotes}

%Pakete für Tabellen
\usepackage{epstopdf}
\usepackage{nicefrac} % Brüche
\usepackage{multirow}
\usepackage{rotating} % vertikal schreiben
\usepackage{colortbl}
\usepackage{mdwlist}

\definecolor{dunkelgrau}{rgb}{0.8,0.8,0.8}
\definecolor{hellgrau}{rgb}{0.0,0.7,0.99}
% Colors for listings
\definecolor{mauve}{rgb}{0.58,0,0.82}
\definecolor{dkgreen}{rgb}{0,0.6,0}

% sauber formatierter Quelltext
\usepackage{listings}
\lstset{numbers=left,
	numberstyle=\tiny,
	numbersep=5pt,
	breaklines=true,
	showstringspaces=false,
	frame=l ,
	xleftmargin=5pt,
	xrightmargin=5pt,
	basicstyle=\ttfamily\scriptsize,
	stepnumber=1,
	keywordstyle=\color{blue},          % keyword style
  	commentstyle=\color{dkgreen},       % comment style
  	stringstyle=\color{mauve}         % string literal style
}

% Biblatex
\usepackage[
backend=biber,
style=numeric,
citestyle=authoryear,
url=false,
isbn=false,
notetype=footonly,
hyperref=false,
sortlocale=de]{biblatex}

%weitere Anpassungen für BibLaTex
\input{skripte/modsBiblatex}


%Bib-Datei einbinden
\addbibresource{literatur/literatur.bib}

% Pfad fuer Abbildungen
\graphicspath{{./}{./abbildungen/}}

%-----------------------------------
% Weitere Ebene einfügen
\input{skripte/weitereEbene}

%-----------------------------------
% Zeilenabstand 1,5-zeilig
%-----------------------------------
\usepackage{setspace}
\onehalfspacing

%-----------------------------------
% Absätze durch eine neue Zeile
%-----------------------------------
\setlength{\parindent}{0mm}
\setlength{\parskip}{0.8em plus 0.5em minus 0.3em}

\sloppy					%Abstände variieren
\pagestyle{headings}

%-----------------------------------
% Abkürzungsverzeichnis
%-----------------------------------
\usepackage[intoc]{nomencl}
\renewcommand{\nomname}{Abkürzungsverzeichnis}
\setlength{\nomlabelwidth}{.20\textwidth}
\renewcommand{\nomlabel}[1]{#1 \dotfill}
\setlength{\nomitemsep}{-\parsep}
\makenomenclature

%-----------------------------------
% Meta informationen
%-----------------------------------
%-----------------------------------
% Meta Informationen zur Arbeit
%-----------------------------------

% Autor
\newcommand{\myAutor}{Alexsandar Petrov}

% Adresse
\newcommand{\myAdresse}{Hordeler Stra\ss e 62 \\ \> \> 44809 Bochum}

% Titel der Arbeit
\newcommand{\myTitel}{Unterstützung von Informationsflüssen des Fahrdienstes in einem Nahverkehrsunternehmen durch ein Groupwaresystem - Eine Anforderungsanalyse}

% Betreuer
\newcommand{\myBetreuer}{Dipl.-Wirtschaftsinf. Dr. Mike Heuser}

% Lehrveranstaltung
\newcommand{\myLehrveranstaltung}{IT-Infrastruktur}

% Matrikelnummer
\newcommand{\myMatrikelNr}{423066}

% Ort
\newcommand{\myOrt}{Essen}

% Datum der Abgabe
\newcommand{\myAbgabeDatum}{21. Januar 2018}

% Semesterzahl
\newcommand{\mySemesterZahl}{3}

% Name der Hochschule
\newcommand{\myHochschulName}{FOM Hochschule für Oekonomie \& Management Essen}

% Standort der Hochschule
\newcommand{\myHochschulStandort}{Standort Essen}

% Studiengang
\newcommand{\myStudiengang}{Wirtschaftsinformatik}

% Art der Arbeit
\newcommand{\myThesisArt}{Seminararbeit}

% Zu erlangender akademische Grad
\newcommand{\myAkademischerGrad}{Bachelor of Science (B. Sc.)}

% Firma
\newcommand{\myFirma}{Bogestra AG}


%-----------------------------------
% PDF Meta Daten setzen
%-----------------------------------
\hypersetup{
    pdfinfo={
        Title={\myTitel},
        Subject={\myStudiengang},
        Author={\myAutor},
        Build=1.1
    }
}

%-----------------------------------
% Umlaute in Code korrekt darstellen
% siehe auch: https://en.wikibooks.org/wiki/LaTeX/Source_Code_Listings
%-----------------------------------
\lstset{literate=
	{á}{{\'a}}1 {é}{{\'e}}1 {í}{{\'i}}1 {ó}{{\'o}}1 {ú}{{\'u}}1
	{Á}{{\'A}}1 {É}{{\'E}}1 {Í}{{\'I}}1 {Ó}{{\'O}}1 {Ú}{{\'U}}1
	{à}{{\`a}}1 {è}{{\`e}}1 {ì}{{\`i}}1 {ò}{{\`o}}1 {ù}{{\`u}}1
	{À}{{\`A}}1 {È}{{\'E}}1 {Ì}{{\`I}}1 {Ò}{{\`O}}1 {Ù}{{\`U}}1
	{ä}{{\"a}}1 {ë}{{\"e}}1 {ï}{{\"i}}1 {ö}{{\"o}}1 {ü}{{\"u}}1
	{Ä}{{\"A}}1 {Ë}{{\"E}}1 {Ï}{{\"I}}1 {Ö}{{\"O}}1 {Ü}{{\"U}}1
	{â}{{\^a}}1 {ê}{{\^e}}1 {î}{{\^i}}1 {ô}{{\^o}}1 {û}{{\^u}}1
	{Â}{{\^A}}1 {Ê}{{\^E}}1 {Î}{{\^I}}1 {Ô}{{\^O}}1 {Û}{{\^U}}1
	{œ}{{\oe}}1 {Œ}{{\OE}}1 {æ}{{\ae}}1 {Æ}{{\AE}}1 {ß}{{\ss}}1
	{ű}{{\H{u}}}1 {Ű}{{\H{U}}}1 {ő}{{\H{o}}}1 {Ő}{{\H{O}}}1
	{ç}{{\c c}}1 {Ç}{{\c C}}1 {ø}{{\o}}1 {å}{{\r a}}1 {Å}{{\r A}}1
	{€}{{\EUR}}1 {£}{{\pounds}}1
}

%-----------------------------------
% Kopfbereich / Header definieren
%-----------------------------------
\pagestyle{fancy}
\fancyhf{}
\fancyhead[C]{-\ \thepage\ -}						% Seitenzahl oben, mittg
%\fancyhead[L]{\leftmark}							% kein Footer vorhanden
\renewcommand{\headrulewidth}{0.4pt}


%-----------------------------------
% Start the document here:
%-----------------------------------
\begin{document}

\pagenumbering{Roman}								% Seitennumerierung auf römisch umstellen
\renewcommand{\refname}{Literaturverzeichnis}		% "Literatur" in
%"Literaturverzeichnis" umbenennen
\newcolumntype{C}{>{\centering\arraybackslash}X}	% Neuer Tabellen-Spalten-Typ:
%Zentriert und umbrechbar

%-----------------------------------
% Titlepage
%-----------------------------------
\begin{titlepage}
	\newgeometry{left=2cm, right=2cm, top=2cm, bottom=2cm}
	\begin{center}
		\textbf{\myHochschulName}\\
		\textbf{\myHochschulStandort}\\
		\vspace{1.5cm}
			\includegraphics[width=3cm]{abbildungen/fomLogo.jpg} \\
		\vspace{1.5cm}
		Berufsbegleitender Studiengang\\
		\myStudiengang, \mySemesterZahl. Semester\\
		\vspace{2cm}
		\textbf{\myThesisArt}\\
		%\textbf{zur Erlangung des Grades eines}\\
		%\textbf{\myAkademischerGrad}\\
		% Oder für Hausarbeiten:
		\textbf{im Rahmen der Lehrveranstaltung}\\
		\textbf{\myLehrveranstaltung}\\
		\vspace{2cm}
		über das Thema\\
		\Huge{\myTitel}\\
		\vspace{0.2cm}
	\end{center}
	\normalsize
	\vfill
	\begin{tabbing}
		Links \= Mitte \= Rechts\kill
		Betreuer: \> \> \myBetreuer\\
		\> \> \\

		Autor: \> \> \myAutor\\
		\> \>  Matrikelnr.: \myMatrikelNr\\
		\> \> \myAdresse\\
		\> \> \\
		Abgabe: \> \> \myAbgabeDatum
	\end{tabbing}
\end{titlepage}

%-------Ende Titelseite-------------

%-----------------------------------
% Sperrvermerk
%-----------------------------------
%\input{kapitel/anhang/sperrvermerk}

%-----------------------------------
% Inhaltsverzeichnis
%-----------------------------------
\setcounter{page}{1}
\tableofcontents
\newpage

%-----------------------------------
% Abkürzungsverzeichnis
%-----------------------------------
\printnomenclature
\newpage
%-----------------------------------
% Abbildungsverzeichnis
%-----------------------------------
\listoffigures
\newpage
%-----------------------------------
% Tabellenverzeichnis
%-----------------------------------
\listoftables
\newpage
%-----------------------------------
% Seitennummerierung auf arabisch und ab 1 beginnend umstellen
%-----------------------------------
\pagenumbering{arabic}
\setcounter{page}{1}
%-----------------------------------
% Kapitel / Inhalte
%-----------------------------------
\input{kapitel/einleitung/einleitung}
\input{kapitel/kapitel_1/kapitel_1}
\input{kapitel/kapitel_2/kapitel_2}
\input{kapitel/fazit/fazit}

%-----------------------------------
% Literaturverzeichnis
%-----------------------------------
\newpage
%\addcontentsline{toc}{section}{Literatur}

\pagenumbering{Roman} %Zähler wieder römisch ausgeben
\setcounter{page}{4}  %Zähler manuell hochsetzen

\printbibliography

% Alternative Darstellung:
% Literaturverzeichnis nach Typ (@book, @arcticle ...) sortiert.
% Dazu die Zeile (\printbibliography) auskommentieren und folgenden code verwenden:

%\printbibheading
%\printbibliography[type=article,heading=subbibliography,title={Artikel}]
%\printbibliography[type=book,heading=subbibliography,title={Bücher}]
%\printbibliography[type=online,heading=subbibliography,title={Webseiten}]

\input{kapitel/anhang/erklaerung}
\end{document}
